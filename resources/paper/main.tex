\documentclass{article}
\usepackage[utf8]{inputenc}
\usepackage{amsmath}

\title{Multimodal Contextual Dialog State Tracking As Bayesian Specific Signal Transduction}
\author{
	Birkner, Joseph \\ \texttt{joseph.birkner@tum.de}
	\and
	Karimi, Negin \\ \texttt{negin.karimi@tum.de}
	\and
	Dolp, Andreas \\ \texttt{andreas.dolp@tum.de}
	\and
	Airiian, Wagram\\ \texttt{wagram.airiian@tum.de}
	\and
	Kharchenko, Alona \\ \texttt{unicorn@roboy.org}
	\and
	Hostettler, Rafael \\ \texttt{rh@gi.ai}
}
\date{April 2019}


\begin{document}

\maketitle

\section{Abstract}

A major challenge in the development of Natural Language Dialog Systems is to determine the intent of a user utterance, and to map the intent of an utterance U to a certain dialog application state T.
While recent work in this area focuses on embedding these variables as Neural-Network generated latent representations, we hypothesize that a symbolic approach to Dialog State tracking might deliver higher utility with reduced development effort:
By observing dialog system state as words out of a formal language over signals in the application context, with application states acting as contextual non-terminals, we set up a basic formal framework for dialog state propagation.
Futhermore, we propose the notion of constraint-based Bayesian state specificity as utility to resolve conflicts between overlapping application states. We implement our system in the open-source library "Ravestate".
Experiments with the implemented system both in text- and speech based scenarios with additional video input show very robust contextual behavior, while operating fully causally explainable and transparently.

\section{Introduction}


\section{Related Work}


\section{State Tracking As Signal Transduction}


\subsection{Terminology}

States $t \in T$ (Processes, Transition Functions, Transducers, Non-Terminals)

Properties $p \in P$  (Data, Channels)

Signals $c \in C$ (Constraints, Chunks)

Spikes $\hat{s_c} \in \hat{S}$

Activations $\hat{a_t} \in \hat{A}$

CausalGroup spike equivalence class $[\hat{s_c}]$.
2
\subsection{Simple Example}

\subsection{Core Protocol}

\subsubsection{$\text{EMIT} (G,C)$}

\subsubsection{$\text{WIPE} (G,C)$}

\subsubsection{$\text{ACQUIRE} (G,A, C)$}

\subsubsection{$\text{WITHDRAW} (G,A, C)$}

\subsubsection{•}



\section{Bayesian Specificity As State Utility}


\section{Experiments}


\section{Conclusion}


\section{Future Work}


\section{References}


\end{document}
